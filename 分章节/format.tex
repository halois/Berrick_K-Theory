% !mode::"TeX:UTF-8"
%!TEX program = xelatex
% +-----------------------------------------------------------------------------
% | File: Berrick $K$-theory
% | Author: Berrick
% |
% | 
% | 
% | Description:
% |     这个是排版源文件,后来分割成format,main,cha0...,ref等文件
% |     遇到一些问题请参考该文档注释掉的东西,正文中可能有错。后期更改是直接在分割后的文件
% | 中直接修改的,于是在本文档中可能出现未修改的内容。
% | 本tex文档的解释权归张浩所有
% +-----------------------------------------------------------------------------
\documentclass[openany,leqno]{book}  

\usepackage{imakeidx}
\usepackage{multicol,multirow} % used for the two-column index 多行多列, 在表格中会用到
\usepackage[bottom]{footmisc}
\usepackage{url}  %这个是加超链接
\usepackage{indentfirst} %首行缩进
\usepackage{amscd,xy,amssymb,amsmath,amsthm,nicefrac,amsfonts,pifont}
\usepackage{mathtools}
\usepackage{tikz}
	\usetikzlibrary{cd}
\usepackage{tikz-cd} %xymatrix包文中用过几处, 后来都是在用tikz-cd来画交换图
\xyoption{all}
\newdir{ >}{{}*!/-5pt/@{>}}
\usepackage{microtype} %microtype 宏包可以改善了单词、字母的间距。它可能做了很多, 但是大部分人察觉不到使用它之后文档的变化。但至少, 加载了 microtype 之后, 文档看起来更好, 也更容易阅读。注意:如果有使用到字体宏包, 需要将 microtype 宏包放在它们的后面, 因为这个宏包对单词、字母的调整和字体是有关的。

\renewcommand\bibname{References} %修改bib的名称

%--一些数学符号, 常用的有\Hom \End \iso(向右箭头上有同构) \id \ker \ima \coker \tor \ext \N \Z \Q \R \mathbb{C}
\newcommand{\Hom}{\mathop{\mathrm{Hom}}} %这个常用
\newcommand{\End}{\mathrm{End}}
\newcommand{\id}{\mathrm{id}} %这个常用
\newcommand{\coker}{\mathrm{coker}}
%\newcommand{\ker}{\mathrm{ker}} %\ker直接输入就好
\newcommand{\tor}{\mathrm{Tor}}
\newcommand{\ext}{\mathrm{Ext}}
\newcommand{\ima}{\mathrm{Im}}
\newcommand{\N}{\mathbb{N}}
\newcommand{\Z}{\mathbb{Z}}
\newcommand{\Q}{\mathbb{Q}}
\newcommand{\R}{\mathbb{R}}
% \newcommand{\C}{\mathbb{C}}  %注销这个的原因是\C 和hyperref宏包冲突。后放弃hyperref,还是有很多冲突,比如页码不对
\newcommand{\F}{\mathbb{F}}

%范畴学常用
\newcommand{\J}{\mathcal{J}}  
\newcommand{\cC}{\mathcal{C}} 
\renewcommand{\top}{\rm top}
\newcommand{\ab}{\mathfrak{Ab}}
\newcommand{\Grp}{\mathfrak{Grp}}
\renewcommand{\mod}{{\rm Mod}}
\newcommand{\Obj}{\mathop{\mathrm{Obj}}}
\newcommand{\sets}{\mathrm{Sets}}
\newcommand*{\colim}{\mathop{\mathrm{colim}}}
\newcommand{\coeq}{\mathrm{coeq}}
\newcommand{\coli}{\colim}

%一些箭头
\newcommand{\iso}{\overset{\cong}{\to}} %isomorphic 这个\to 可以换成\longrightarrow 变长
\newcommand{\weq}{\overset{\sim}{\to}} %表示weak equivalence
\newcommand{\cofi}{\rightarrowtail} %带尾巴的箭头, 单射或者余纤维cofibration
\newcommand{\fib}{\twoheadrightarrow} %满射 或者纤维化fibration
\newcommand{\defo}{\overset{\sim}{\fib}} %这个没用过
\newcommand{\DA}{\Delta} %一般也不用

\def\idem{\mathrm{Idem}}
\renewcommand{\inf}{\rm inf}
\newcommand{\nil}{\rm nil}
\newcommand{\ass}{\mathfrak{Ass}}
\newcommand{\der}{\mathrm{Der}}

%------这个部分包含了计数器,使得定理,引理等按照同一顺序编号
\newtheorem{theorem}{Theorem}[chapter]
\renewcommand\thetheorem{\arabic{chapter}.\arabic{theorem}} 
\newtheorem{lemma}[theorem]{Lemma}
\renewcommand\thelemma{\arabic{chapter}.\arabic{lemma}} 
\newtheorem{prop}[theorem]{Proposition}
\renewcommand\theprop{\arabic{chapter}.\arabic{prop}} 
\newtheorem{ex}[theorem]{Example}
\renewcommand\theex{\arabic{chapter}.\arabic{ex}} 
\newtheorem{corollary}[theorem]{Corollary}
\renewcommand\thecorollary{\arabic{chapter}.\arabic{corollary}} 
\renewcommand\theequation{\thetheorem}

\makeindex
\setlength{\textwidth}{15cm} \setlength{\textheight}{22cm}
\setlength{\oddsidemargin}{0.5cm} \setlength{\topmargin}{0cm}
\setlength{\evensidemargin}{0.5cm} \setlength{\topmargin}{0cm}