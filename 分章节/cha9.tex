%---------------第九章------------------%
\chapter{Quillen's higher $K$-groups} % (fold)
\label{cha:9quillen_s_higher_k_groups}

The intention is to define the higher $K$-groups of a ring $A$, $K_i(A)$ $(i\geqslant 1)$, as the composition of covariant functors
\[K_i\colon   A\mapsto GL(A) \mapsto BGL(A) \mapsto BGL(A)^+\mapsto \pi_i(BGL(A)^+)\]
As a matter of procedure though, we should first use (8.6) to check that this agrees with two established case, $K_1$ and $K_2$. The former is quite obvious, since by (1.11) we have that
\[\pi_1(BGL(A)^+=GL(A)/P(GL(A))=GL(A)/E(A)=K_1(A).\]
For the latter, because by (8.6) $\pi_2(BGL(A)^+)$ is the kernel of the universal central extension of $P(GL(A))=E(A)$, it suffices to show:
\begin{theorem}
The central extension (3.7)
\[K_2(A)\rightarrowtail St(A)\twoheadrightarrow E(A)\]
is the u.c.e. of $E(A)$.
\end{theorem}
\begin{proof}
We have to show that $St(A)=SP(GL(A))$, in other words, that $St(A)$ is superperfect. Certainly $H_1(St(A))=0$(1.3), leaving only the vanishing of $H_2(St(A))$. Now $St(A)$ is defined as the direct limit of the group $St_n(A)$, and the homology perserves direct limits. So we show that, for $n\geqslant 5$, $\iota\colon  St_{n-1}(A)\rightarrow St_n(A)$ has $H_2(\iota)$ trivial. To do this we shall check any central extension $K \cofi G \overset{\psi}{\fib} St_n(A)$ splits over $St_{n-1}(A)$, whence, from (8.3), $H^2(\iota)=0\colon   H^2(St_n(A);K)\rightarrow H^2(St_{n-1}(A);K)$. Then by the universal coefficient theorem [38(5.5.3)] when $K=H_2(St_n(A))$,
\[H_2(\iota)^*=0\colon   \Hom (H_2(St_n(A)), H_2(St_n(A)))\rightarrow \Hom (H_2(St_{n-1}(A)), H_2(St_n(A))),\]
giving the result because $H_2(\iota)=H_2(\iota)^*(\id)$

Note that, since $K$ is central in $G$, we have, for any $g_1,g_2\in G, [Kg_1, g_2]=[g_1, g_2]$, where we recall that $[g_1, g_2]=g_1g_2g_1^{-1}g_2^{-1}=[g_2, g_1]^{-1}$. There further facts are worth featuring.

{\em In any group $G$, for $u,v,w\in G$,}
\begin{equation*}
\begin{array}{rlc}
& [[u,v],w] \\
(a) & =[u,v][w,v][v,wu],\\
(b) &=[[u,v],[w,v]][[w,v],u] &\text{if } [u,w]\in Z(G), \\
(c) &=1 & \text{if also } [v,w]\in Z(G).
\end{array}
\end{equation*}
These follow $(a)$ by outright multiplication, and $(b)$ by substitution for $[v,wu]=[v,uw]$ in $(a)$. 

Therefore denote by $g_{ij}^a$ the coset $\psi^{-1}(x_{ij}^a)$ of $K$ in $G$ ($i,j\leqslant n; a\in A$). Define (uniquely) $\sigma\colon   St_{n-1}(A)\rightarrow G$ by means of the function
\[x_{ij}^a \mapsto [g_{in}^1,g_{nj}^a]\]
on generators (so $i,j<n$). Clearly $\psi \circ \sigma =\id$ as required. What has to be checked is that the three sets of relations in $St_{n-1}(A)$ are respected by $\sigma$, in order that it be a homomorphism. First, whenever $i,j,k,l\leqslant n$, choose $h\notin \{i,j,k,l\}, h\leqslant n$ (possible, since $n\geqslant 5$). Then because $g_{ij}^a=K[g_{ih}^1,g_{hj}^a]$ we deduce from $(c)$ above that 
\begin{equation*}
\begin{array}{rcr}
(d) &[g_{ij}^a,g_{kl}^b]=1 & \text{if } i\neq l, j\neq k
\end{array}
\end{equation*}
As
\[\sigma(1)=\sigma(x_{12}^0)=[g_{1n}^1,g_{n2}^0]=[g_{1n}^1,K]=1,\]
this implies in particular that, for $i\neq l, j\neq k$,
\[[\sigma(x_{ij}^a),\sigma(x_{kl}^b)]=1=\sigma([x_{ij}^a,x_{kl}^b]).\]
The next argument begins with $(a)$ in the form
\[[v,w][v,u]=[v,wu][w,[v,u]^{-1}],\]
and conclude with $(d)$ $\cdots$
\begin{equation*}
\begin{array}{rrl}
 &\sigma(x_{ij}^a)\sigma(x_{ij}^b)&=[g_{in}^1,g_{nj}^ag_{nj}^b][g_{nj}^a,[g_{in}^1,g_{nj}^b]^{-1}] \\
 & &=[g_{in}^1,g_{nj}^{a+b}][g_{nj}^a,g_{ij}^{-b}]\\
 (e)& &=\sigma(x_{ij}^{a+b}). 
\end{array}
\end{equation*}
Finally, for $i\neq k$, as an application of $(b)$, $(d)$ and $(e)$ in turn, $\cdots$
\begin{equation*}
\begin{array}{rl}
[\sigma(x_{ij}^a),\sigma(x_{jk}^b)] & =[[g_{in}^1,g_{nj}^a],g_{jk}^b]\\
& =[[g_{in}^1,g_{nj}^a],[g_{jk}^b,g_{nj}^a]][[g_{jk}^b,g_{nj}^a],g_{in}^1]\\
& =[g_{ij}^a,g_{nk}^{-ab}][g_{in}^1,g_{nk}^{-ab}]^{-1}\\
& =(\sigma(x_{ik}^{-ab}))^{-1}\\
& =\sigma(x_{ik}^{ab}).
\end{array}
\end{equation*}
Hence $\sigma$ preserves the relations in $St_{n-1}(A)$ and is a homomorphism after all, completing the proof. We remark that with a little extra tedium we could have extended $\sigma$ over $St_n(A)$ (proving that $St_n(A)$ too is superperfect for $n\geqslant 5$); however this would have bogged us down in a check that $\sigma$ was well-defined over enlarged domain.
\end{proof}
This gives us the ``all-clear" to make the longed-for definition:
\[K_i(A)=\pi_i(BGL(A)^+), i\geqslant 1,\]
Not only does (9.1) confirm that $\pi_2(BGL(A)^+)$ is isomorphism to Milnor's $K_2(A)$ of Chapter 3, but because it also identifies $(BGL(A))_3$ as $BSt(A)$ we may apply (8.6) and (7.8).
\begin{theorem}
  \[K_1(A)=H_1(GL(A)),\]
  \[K_2(A)=H_2(E(A)),\]
  \[K_3(A)=H_3(St(A)),\]
  and
  \[K_2(A)\twoheadrightarrow H_2(GL(A)),\]
  \[K_3(A)\twoheadrightarrow H_3(E(A)),\]
  \[K_4(A)\twoheadrightarrow H_4(St(A)).\]
\end{theorem}
The last two epimorphisms of (9.2) follow from (7.8), while the surjection $\pi_2(BGL(A)^+)\twoheadrightarrow H_2(BGL(A)^+)$ is a consequence of the isomorphism in dimension $1$, by the generalized Hurwicz argument again (p. 59).

Now that we have been more forthcoming about the space to which we are applying the plus-construction, we can also be more specific about the construction itself. With extra care we can minimize the number of $2-$ and $3-$cells added to $BGL(A)$ in (5.1) to form $BGL(A)^+$. Further, specification of actual attaching maps for these cells gives a determination of the space $BGL(A)^+$ itself, rather than, as above, of its homotopy type under $BGL(A)$. (This of course assumes a definite choice of model for the classifying space, such as Milnor's quotient of the infinite join.) Undoubtedly, this would be a tiresome procedure if it had to be performed separately for each ring $A$. Fortunately however, the category $\mathbb{R}ing$\index{Ring@$\mathbb{R}ing$} has an initial object; so once the cells have been adjoined to $BGL(\mathbb{Z})$ their addition to $BGL(A)$ follows from the commuting square
\refstepcounter{theorem}
\begin{equation}
 \xymatrix{
BGL(\mathbb{Z}) \ar[r] \ar[d] & BGL(\mathbb{Z})^+ \ar[d]\\
BGL(A) \ar[r] & BGL(A)^+
} 
\end{equation}
%编号要改
when it is co-Cartesian, which, by the comment after (\ref{5.11}), occurs precisely when $E(A)$ is the normal closure of $\textrm{Im}(E(\mathbb{Z}\longrightarrow E(A))$. Thus, as $E(A)$ is the homomorphic image of $St(A)$, the achievement of these goals hinges on the following technical lemma.
\begin{lemma}
  In $St(\mathbb{Z})$, let $\omega=u_{12}u_{13}^{-1}$ where $u_{ij}=x_{xj}^1x_{ji}^{-1}x_{ij}^1$. Then $St_n(A)$ is the normal closure of $\omega$, provided $4\leqslant n \leqslant \infty$.
\end{lemma}
\begin{proof}
  The proof is an exercise in $St(A)$-manipulation. First show that, for $j\neq 1,k,$
  \[u_{1j}x_{1k}^au_{1j}^{-1}=x_{jk}^{-a}.\]
  Thus, for $k\neq 1,2,3,$
  \[[\omega, x_{3k}^a]=u_{12}x_{2k}^au_{12}^{-1}x_{3k}^{-a}=x_{2k}^ax_{3k}^{-a}.\]
  So (a) of the proof of (9.1) (with $v=x_{h2}^1, w=x_{2k}^a, u=x_{3k}^{-a}$) implies that
  \[[x_{h2}^1, [\omega, x_{3k}^a]]=x_{hk}^a,\]
  provided $h\neq 2,k$. If $h=2$, then substitution of $v=x_{23}^{-1}$ instead yields that
  \[[x_{23}^{-1}, [\omega, x_{3k}^a]]=x_{2k}^a.\]
  This means that the normal closure of $\omega$ contains $x_{hk}^a$ whenever $k\neq 1,2,3,h$. A similar argument shows that it also contains $x_{kh}^a$; yet for $n\geqslant 4$ every standard generator of $St_n(A)$ either takes (at least) one of these two forms, or is a commutator of two such.
\end{proof}
\begin{corollary}
  The space $BGL(A)^+$ may be chosen so that the square (9.3) is co-Cartesian.
\end{corollary}
\begin{corollary}
  $BGL(A)^+$ may be formed from $BGL(A)$ by the adjunction of a single $2$-cell and $3$-cell to $BGL_3(A)$, and, for $n\geqslant 4$, $BGL(A)^+=BGL(A)\cup_{BGL_n(A)}BGL_n(A)^+$.
\end{corollary}
The element $u_{12}$ above was used in the proof of (1.9). Note too that $t=u_{12}$ is conjugate by $x_{12}^1$ to $s=x_{21}^{-1}x_{12}^2$, where $s^4$ generate $K_2(\mathbb{Z})$ (p.26). Moreover, in $E(\mathbb{Z})$,
\[\varphi(s^2)=\varphi(t^2)= \begin{pmatrix}-1 &0 \\ 0 & -1\end{pmatrix},\]
so that $s^2 t^{-2} \in \ker(\varphi\colon   St(\mathbb{Z})\longrightarrow E(\mathbb{Z}))=K_2(\mathbb{Z})=Z(St(\mathbb{Z}))$.
 By putting $u=w=s^2$ and $v=x_{12}^1$ in (a), (c) of the proof of (9.1), we obtain the triviality of $[s^2, x_{12}^1]^2=(s^2 t^{-2})^2$. Hence $s^4 t^{-4}=(s^2 t^{-2})^2 [t^2 s^{-2}, t^2]$ is also trivial, making $t^4$ an alternative description of the generator of $K_2(\mathbb{Z})$.
\section*{SUMMARY OF RESULTS }
Here is a very rapid review of the highlights of what is known about the size of the groups $K_i(A) (i\geqslant 0)$. In general I do not even hint at the methods of proof; suffice it to say they vary widely, both from each other and from those of this book.

In the course of the next two chapters we shall see that $BGL(A)^+$ has the structure of the space of loops an another space, and therefore admits a multiplication map
\[BGL(A)^+ \times BGL(A)^+ \longrightarrow BGL(A)^+ \]
corresponding to composition of loops. According to Milnor and Moore [32], the rational homotopy type ($\pi_*(X)\otimes \mathbb{Q}$) of such a space, $X$ say, is determined by the rational homology, as follows. An element $x\in H_*(x,\mathbb{Q})$ is said to pe {\em primitive} if
\[\Delta_*(x)=x\otimes 1+1\otimes x,\]
where from (3.11) we recall that $\Delta\colon   X \longrightarrow X\times X$ is the diagonal map, and $x\mapsto x\otimes 1$ is induced by $\textrm{in}_L\colon   x\mapsto (x,\textrm{pt})$, etc.\ 
\begin{theorem}
  $K_i(A)\otimes \mathbb{Q}$ is isomorphism to the subgroup of primitive element in homologys\index{primitive element in homology} of $H_i(GL(A);\mathbb{Q})$.
\end{theorem}
Here, the homology of $BGL(A)^+$ has been identified with that of $BGL(A)$ and so the group $GL(A)$. Now let $A$ be the ring of {\em algebraic integers}\index{algebraic integers} (= roots of monic polynomials with integer coefficients) in a {\em number field}\index{number field} (=finite $\mathbb{Q}$-extension). In this situation, Borel [5] has successfully performed the homological computation. (The statement uses the Kronecker delta again.)
\begin{theorem}
  If $A$ is the ring of algebraic integers in a number field $F$ with $r_1$ real and $r_2$ complex places\index{place in a number field} (that is, $F\otimes_{\mathbb{Q}}\mathbb{R}=\mathbb{R}^{r_1}\times \mathbb{C}^{r_2}$), then $K_{2j}(A)$ has rank $\delta_{j,0}$, and $K_{2j+1}(A)$ has rank $r_2$ for $j$ odd and $r_1+r_2-\delta_{j,0}$ for $j$ even.
\end{theorem}
A further constraint on the size of $K_i(A)$ here was provided by Quillen [41, pp. 179-198].
\begin{theorem}
  For $A$ as in (9.8), $K_i(A)$ is a finitely generated group.
\end{theorem}
So all eyes turn to the order of the finite subgroup of torsion elements in $K_i(A)$. In this connection, Lichtenbaum\index{Lichtenbaum conjecture} [42, pp. 489-501] has conjectured the ratio of order of $K_{2j}(A)$ ($j$ odd) to that of $K_{2j+1}(A)$ to be (up to $2$-torsion) the value of the zeta-function\index{zeta-function} on $F$ at $-j$, when $r_2 =0$. This conjecture was in part inspired by Quillen's computation in [33] of the $K$-theory of the finite $\mathbb{F}_q$.
\begin{theorem}
  For $j\geqslant 1$, $K_{2j}(\mathbb{F}_q)=0$, and $K_{2j-1}(\mathbb{F}_q)$ is cyclic of order $q^j -1$.
\end{theorem}
Impressive support for the conjecture in general has been provided by Harris and Segal [16]. (See also Browder [8].) However, the specific case to arouse most excitement is of course that where $A=\mathbb{Z}, F=\mathbb{Q}$. Now the zeta-function at $-(2n-1)$ assumes that value $-\frac{B_n}{2n}$, the {\em Bernoulli number} $B_n$\index{Bernoulli number@Bernoulli number, $B_n$} being $(2n)!$ times the coefficient of $t^{2n}$ in the power series expansion of $\frac{t}{e^t-1}$.

More ambitious versions of the conjecture relate the order of $K_{4n-2}(\mathbb{Z})$, $K_{4n-1}(\mathbb{Z})$ to the numerator and denominator respectively of this fraction. Soul\'{e} [37] has obtained partial confirmation here, and his results have recently been extended by Dwyer and Friedlander (to appear).

The denominator of $\frac{B_n}{4n}$ is familiar to homotopy-theorists as the order of the (cyclic) image of the $J$-homomorphism\index{J-homomorphism@$J$-homomorphism} in homotopy groups of sphere, which image Quillen [43, pp. 182-188] shows maps injectively to the homotopy groups of $BGL(\mathbb{Z})^+$. Hence $K_{4n-1}(\mathbb{Z})$ contains an element of order equal to the denominator of $\frac{B_n}{4n}$ ($=\frac{1}{24}$ for $n=1$,$\frac{1}{240}$ for $n=2$, etc.\ ). Moreover, this generates a cyclic summand when $n$ is even. For $n$ odd, it need not generate a summand, as Lee and Sczczarba revealed in their paper. {\em The group $K_3(\mathbb{Z})$ is cyclic of order forty-eight} [27]. Further torsion results appear in [7],[28],[36],[37]; however, the comparison with knowledge of stable homotopy groups of sphere points up how rudimentary is our standing of $K_*(\mathbb{Z})$.

% chapter quillen_s_higher_k_groups (end)