\chapter*{Preface}
\label{cha:0preface}
Two decades ago, the term ``algebraic $K$-theory" had not even been born --- the subject was still in embryo. A decade later, there was no doubt of existence; it was uniqueness which seemed at risk as seemingly distinct competing definitions tumbled forth. Since then there has been much reconciliation and development. It is now unquestionably a subject in its own right, clearly flourishing.  

Yet in certain respects it has grown too fast. The wider mathematical community had found it hard to keep up with the pace, as anyone trying to find the AMS (1980) classification for the subject will appreciate. (There has, however, been significant acknowledgement of its achievements, most notably in the award of a 1978 Fields Medal to one of its exponents, D.~Quillen.) In particular,  it has become very difficult for ``outsider" (whether established Mathematicians in other areas, or graduate students wishing to enter the fray) to gain access to the material which must be mastered before current research can be studied. Nobody has written an introduction to the subject since its quite early days. A new one is sorely needed. In addition, Quillen's fundamental tool for the most-travelled path to higher algebraic $K$-theory, the plus-construction, now receives attention from geometric topologists and homotopy theorists. Again, there is a place for a thorough, up-to-date treatment of the basics. 

This book is a modest attempt to meet those needs. In 1976 I gave an eight-lecture graduate course at Oxford, much of its content deriving from (a small portion of) a course Quillen delivered at M.I.T. while I was visiting there in 1974--1975. The next few years saw a number of developments, so that by 1981 a repeat course at Oxford comprised sixteen lectures. These now constitute the core of the book, although the inevitable inflation has resulted in only one chapter(the eighth) resembling lecture-sized length. 

I have in mind as likely reader an algebraist, topologist or algebraic number-theorist requiring first of all a little motivation before confronting the technicalities. The first three chapters therefore emphasize the history of the topic and its relationship to its ``older sibling", topological $K$-theory. In chapter 4 the real work begins, in the shape of an introduction to the plus-construction. There is, alas, no escaping the reader's need for a little topological familarity here, but I have tried to minimize this. The pleasent surprise is how much algebra, in the form of fundamental group-actions on homotopy and homology groups, lies at the heart of this approach. As a result, the discussion (in Chapter 9 and 11) of seemingly topological properties of the topological space $BGL(A)^+$ is in large part algebraic. For me, an especially satisfying aspect of the treatment is the way in which the same lemma(3.11) in homology of groups is revealed as lying behind both the de-looping of $BGL(A)^+$ in chapter 11 and the passage from equivariant $A$-representations to $KA$-theory exploited in Chapter 12 and 13. 

Throughout I have tried to complement, rather than reproduce, established references. $K$-theory veterans, as well as novices, should therefore find something here to interest them. By making the plus-construction the central theme of the work, I have automatically ruled out chances of definitiveness: some results which I state for completeness' sake have been established by quite different methods. The reader should be aware that this is only {\em an} approach to the subject. I hope it encourages him/her also to contemplate others. 

The appearance of this book owes much to a number of people. Prominent have been I.\,M.~James at Oxford and G.\,E.\,H.~Reuter at Imperial College, London, my two Heads of department over the relevant years. E.~Dror(Jerusalem/Oxford) had helpful comments on a draft of much of the work. My erstwhile algebra colleagues at I.C.L. made continual contributions. In particular, M.\,E.~Keating did a marvellous job srutinizing drafts; I feel sure his efforts have made the presentation more accessible to algebraists. Last (chronologically), Mrs. M.~Robertson can take credit for the outstanding {\em typographical} clarity of what follows.


\noindent Oxford\\
September, 1981. \\
{\begin{flushright}
  Jon Berrick.
\end{flushright}} 
% chapter preface (end)